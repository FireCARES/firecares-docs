\documentclass[12pt,oneside]{book}
\usepackage{times,mathptmx}
\usepackage[pdftex]{graphicx}
\usepackage{calc}
\usepackage{tabularx,ragged2e,booktabs,caption,subcaption}
\usepackage{array}
\newcolumntype{L}[1]{>{\raggedright\let\newline\\\arraybackslash\hspace{0pt}}m{#1}}
\newcolumntype{C}[1]{>{\centering\let\newline\\\arraybackslash\hspace{0pt}}m{#1}}
\newcolumntype{R}[1]{>{\raggedleft\let\newline\\\arraybackslash\hspace{0pt}}m{#1}}
\usepackage{multirow}
\usepackage{multicol}
\usepackage{tocloft}
\usepackage{xcolor}
\usepackage{color,soul}
\usepackage{amsmath}
\definecolor{linknavy}{rgb}{0,0,0.50196}
\definecolor{linkred}{rgb}{1,0,0}
\definecolor{linkblue}{rgb}{0,0,1}
\definecolor{darkorange}{rgb}{0.81,0.52,0}
\definecolor{fc_orange}{rgb}{0.94,0.59,0.14}
\definecolor{brown}{rgb}{0.56,0.36,0}
\definecolor{fc_blue}{rgb}{0.27,.36,0.48}
\usepackage{float}
\usepackage{graphpap}
\usepackage{rotating}
\usepackage{graphicx}
\usepackage{geometry}
\usepackage{relsize}
\usepackage{ltablex}
\usepackage{longtable}
\usepackage{lscape}
\usepackage{amssymb}
\usepackage{makeidx} % Create index at end of document
\usepackage[nottoc,notlof,notlot]{tocbibind} % Put the bibliography and index in the ToC
\usepackage{lastpage} % Automatic last page number reference.
\usepackage[T1]{fontenc}
\usepackage{enumerate}
\usepackage{upquote}
\usepackage{moreverb}
\usepackage{xfrac}
\usepackage{cite}
\usepackage{tikz}
% \usepackage{subfig}
% \usepackage{caption}
\usepackage[toc,page]{appendix}
\usepackage{notoccite}
\usepackage{placeins}

\usepackage{titlesec}
\titleformat{\chapter}[hang] 
{\color{fc_blue}\normalfont\huge\bfseries}{\chaptertitlename\ \thechapter}{1em}{}[\titlerule]
\titlespacing*{\chapter}{0pt}{-30pt}{20pt}

\titleformat*{\section}{\normalfont\Large\bfseries\color{fc_blue}}
\titleformat*{\subsection}{\normalfont\large\bfseries\color{fc_blue}}

\newcommand{\nopart}{\expandafter\def\csname Parent-1\endcsname{}} % To fix table of contents in pdf.

\usepackage{siunitx}
\sisetup{
    detect-all = true,
    input-decimal-markers = {.},
    input-ignore = {,},
    inter-unit-product = \ensuremath{{}\cdot{}},
    multi-part-units = repeat,
    number-unit-product = \text{~},
    per-mode = fraction,
    separate-uncertainty = true,
}

\usepackage{listings}
\usepackage{textcomp}
\definecolor{lbcolor}{rgb}{0.96,0.96,0.96}

\usepackage[pdftex,
        colorlinks=true,
        urlcolor=fc_orange,     % \href{...}{...} external (URL)
        citecolor=fc_orange,     % citation number colors
        linkcolor=fc_orange,    % \ref{...} and \pageref{...}
        pdfproducer={pdflatex},
        pdfpagemode=UseNone,
        bookmarksopen=true,
        plainpages=false,
        verbose]{hyperref}

\renewcommand{\cftchapfont}{\hypersetup{linkcolor=fc_blue}}
\renewcommand{\cftsecfont}{\hypersetup{linkcolor=black}}
\renewcommand{\cftsubsecfont}{\hypersetup{linkcolor=black}}
\renewcommand{\cftsubsubsecfont}{\hypersetup{linkcolor=black}}


\setlength{\textwidth}{6.5in}
\setlength{\textheight}{9.0in}
\setlength{\topmargin}{0.in}
\setlength{\headheight}{0.pt}
\setlength{\headsep}{0.in}
\setlength{\parindent}{0.0in}
\setlength{\itemindent}{0.25in}
\setlength{\oddsidemargin}{0.0in}
\setlength{\evensidemargin}{0.0in}
% \setlength{\leftmargini}{\parindent} % Controls the indenting of the "bullets" in a list
\setlength{\cftsecnumwidth}{0.45in}
\setlength{\cftsubsecnumwidth}{0.5in}
\setlength{\cftfignumwidth}{0.45in}
\setlength{\cfttabnumwidth}{0.45in}
\setlength{\parskip}{1em}

% \newcolumntype{L}{>{\centering\arraybackslash}m{4cm}}

\floatstyle{boxed}
\newfloat{notebox}{H}{lon}
\newfloat{warning}{H}{low}

\newenvironment{conditions}
  {\par\vspace{\abovedisplayskip}\noindent\begin{tabular}{>{$}l<{$} @{${}={}$} l}}
  {\end{tabular}\par\vspace{\belowdisplayskip}}


% Rename chapter headings
\renewcommand{\chaptername}{}
\renewcommand{\bibname}{References}

\usepackage{tikz}
\usetikzlibrary{calc}
\usepackage{enumitem}

\usepackage{fancyhdr}
\pagestyle{fancy}
\lhead{}
\rhead{}
\chead{}
\renewcommand{\headrulewidth}{0pt}

\newdimen\longline
\longline=\textwidth\advance\longline-4cm

\def\LayoutTextField#1#2{#2} % override default in hyperref

\def\lbl#1{\hbox to 4cm{#1\dotfill\strut}}%
\def\labelline#1#2{\lbl{#1}\vbox{\hbox{\TextField[name=#1,width=#2]{\null}}\kern2pt\hrule}}

\def\q#1{\hbox to \hsize{\labelline{#1}{\longline}}\vskip1.4ex}

% \usepackage{draftwatermark}
% \SetWatermarkText{DRAFT}
% \SetWatermarkScale{1}

\begin{document}
\pagenumbering{gobble}

\bibliographystyle{unsrt}
%\pagestyle{empty}

\frontmatter

\begin{minipage}{1.0\textwidth}
\pagecolor{fc_blue}
\pagenumbering{gobble}

\vspace{1cm}
\centering
\includegraphics[width=2.5in]{Figures/firecares-header-logo_2}
\noindent\makebox[\linewidth]{\color{white}\rule{.5\paperwidth}{0.6pt}}

\Huge \color{fc_orange} FireCARES Data Sets \\
\LARGE \color{fc_orange} Data Elements 

\vspace{0.5cm}

\large{\color{white} Version: 1.5  \hspace{1cm} Compilation Date: \today \\}

\vspace{4cm}

{\Large \em \color{fc_orange}Analyze how fire department resources \\
are deployed to match a community's risks. \\
}
\end{minipage}


\frontmatter
\pagecolor{white}

% \newpage
% \hspace{5in}
% \newpage

\pagestyle{fancy}{
\fancyhf{}
\fancyhead[]{%
   \begin{tikzpicture}[overlay, remember picture]%
   \fill[fc_blue] (current page.north west) rectangle ($(current page.north east)+(0,-.5in)$);
   \node[anchor=north west, text=white, font=\Large\scshape, minimum size=1in, inner xsep=5mm] at (current page.north west) {};
   \end{tikzpicture}
}
\fancyfoot[C]{
   \begin{tikzpicture}[overlay, remember picture]%
   \fill[fc_orange] (current page.south west) rectangle ($(current page.south east)+(0,.5in)$);
   \node[anchor=south west, text=black, minimum size=.5in] at (current page.south west) {\thepage};
   \end{tikzpicture}
}}


\fancypagestyle{plain}{
\fancyhf{}%
% \fancyfoot[C]{\thepage\ of \pageref{LastPage}}%
\fancyhead[]{
   \begin{tikzpicture}[overlay, remember picture]% 
   \fill[fc_blue] (current page.north west) rectangle ($(current page.north east)+(0,-.5in)$);
   \node[anchor=north west, text=white, font=\Large\scshape, minimum size=1in, inner xsep=5mm] at (current page.north west) {};
   \end{tikzpicture}
}
\fancyfoot[C]{
   \begin{tikzpicture}[overlay, remember picture]%
   \fill[fc_orange] (current page.south west) rectangle ($(current page.south east)+(0,.5in)$);
   \node[anchor=south west, text=black, minimum size=.5in] at (current page.south west) {\thepage};
   \end{tikzpicture}
}}


\pagenumbering{roman}

\newpage

\cleardoublepage


% \phantomsection

\renewcommand*\contentsname{\color{fc_blue}Contents}
\tableofcontents

\hypersetup{ 
    linkcolor=fc_orange,         % moved after \tableofcontents
    filecolor=fc_orange,      % 
    urlcolor=fc_orange,           %
}    

\newpage
\mainmatter

\chapter{Acknowledgment}

The Fire$--$Community Assessment Response Evaluation System, known as FireCARES, is a `big data' analytical system providing important information to fire service and community leaders about their local fire department and the risk environment in which firefighters are called to respond. FireCARES includes more than a decade of research on structure fires and related injuries and death, as well as building footprints, housing and mobile home units, public health and census data, and vulnerable populations. FireCARES combines large sets of data from various sources to `tell the story' of a fire department in regard to its risk environment, resource capacity, and overall capability to respond to emergency
incidents.

FireCARES$--$The Community Assessment/Response Evaluation System was developed through a Partnership of Fire Service Organizations funded by DHS/FEMA Assistance to Firefighters Grants. The same team of researchers, academics, and fire service professionals who conducted the landmark NIST Residential Fireground Experiments (NIST TN 1661)~\cite{NIST:Residential} and the NIST High-Rise Fireground Experiments (NIST TN 1797)~\cite{NIST:HighRise} built FireCARES. FireCARES partners include NIST, the International Association of Fire Chiefs (IAFC), the Metropolitan Fire Chiefs Association, the International Association of Fire Fighters (IAFF), the Commission on Fire Accreditation International (CFAI-Risk), Underwriters Laboratory Firefighter Safety Research Institute (UL FSRI), the Urban Institute, the University of Texas at Austin, and Worcester Polytechnic.

The original datasets contained in FireCARES were procured and are maintained by the team of experts that developed the system. Every fire department in the United States has a dedicated page in FireCARES. Fire Departments regularly contribute additional data to their respective pages expanding the data assets available. Although we make every effort to accurately represent information in FireCARES, source data used in determining performance scores, community assessments and safe grades may at times be inaccurate due to the quality and or quantity of underlying data sources and could adversely affect these metrics. With the continued efforts of data scientist, developers, and fire service experts, FireCARES is becoming the premier dataset in the fire service. 

\chapter{Model Input Data Elements}

\section{Performance Score}


\begin{table}[H]
  \centering
  \caption{Performance Score Elements}
  \label{tab:psore_data}
  \begin{tabular}{lll}
    \toprule[1.5pt]
    Data Element          & NFIRS Name      & Usage                         \\
    \midrule
    Fire spread category  & fire\_sprd      & Inferring burned area         \\
    Property use          & prop\_use       & Isolating residential fires   \\
    Room/building areas   & N/A             & Inferring burned area         \\
                          &                 &                               \\
                          &                 &                               \\
    Parcel hazard level   & N/A             & Determining whether to include stair climb draws \\
    Fire department       & fd\_id          & Determining which department to score \\
    Fire location         & N/A             & Determining parcel hazard level \\
    \bottomrule[1.25pt]
  \end{tabular}
  \label{Table:Exp1Interventions}
\end{table}

Add text on comments.

\clearpage

\section{Risk Model}
The following data are needed at a fine spatial resolution:

\begin{itemize}
  \item Population Data
  \item Population breakdowns (by age, sex, race?, etc.)
  \item Economic Condition (e.g., household income, employment, etc).
  \item Housing Data
    \begin{itemize}
      \item Housing units by risk categories
      \item Other relevant breakdowns (e.g., single-family residence, multi-family residences, mobile homes, age, owner v. renter, vacancy rate etc.).
    \end{itemize}
  \item Health data
    \begin{itemize}
      \item Smoking prevalence
      \item For EMS Estimates, any other health survey data would be useful.
    \end{itemize}
  \item Fire Data
    \begin{itemize}
      \item Location
      \item Fire Type
      \item Property Use (to identify risk categories)
      \item Casualties
      \item Fire spread / Fire size
    \end{itemize}
\end{itemize}




\chapter{Existing Public Data Elements }
This section includes the lists of data assets available in the FireCARES Database.
\begin{itemize}[noitemsep]
\item Fire Incidents
  \begin{itemize}[noitemsep]
  \item Jurisdiction geopolitical boundary
  \item Call volume
    \begin{itemize}[noitemsep]
    \item 14 years geocoded NFIRS fire data
    \item 4 years geocoded NFIRS EMS data
    \end{itemize}
  \item Census tract geographic layer
  \end{itemize}
\item Jurisdiction Summary Description
  \begin{itemize}[noitemsep]
  \item Department Type
  \item NFPA Region
  \item FDID
  \item State
  \item Phone/Fax
  \item Website link
  \item Population Protected
  \item Community Size
  \item Structure count
    \begin{itemize}[noitemsep]
    \item By community/structure hazard level1
    \item By census tract/structure hazard level
    \end{itemize}
  \end{itemize}
\item Station detail
  \begin{itemize}[noitemsep]
  \item Station list/locations
  \item Apparatus
  \item Staffing (where available)
  \item First due area (where available)
  \item GIS service map – 4, 6, and 8-minute travel times
  \end{itemize}
\item Local Department Data Assets (where available)
  \begin{itemize}[noitemsep]
  \item Inspection Data
  \item Hydrants
  \item First due area by station
  \end{itemize}
\end{itemize}


\chapter{Contact Information}

Mail or email data use request forms to the address below.

For further assistance contact:\\

Dr. Lori Moore-Merrell, President \& CEO \\
International Public Safety Data Institute \\
c/o 4795 Meadow Wood Ln, Suite 100, \\
Chantilly, VA 20151 \\
lori@i-psdi.org \\
202-549-5080 \\


Tyler Garner, Lead Developer \\
garnertb@prominentedge.com \\
434-841-3803 \\


\bibliography{firecares}


\end{document}
